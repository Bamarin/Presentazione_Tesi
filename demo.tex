%! TEX program = xelatex
\documentclass[10pt]{beamer}


\usetheme[progressbar=frametitle]{metropolis}
\setsansfont[BoldFont={Fira Sans}]{Fira Sans Light}
\setmonofont{Fira Mono}
\usepackage{appendixnumberbeamer}

\usepackage{booktabs}
\usepackage[scale=2]{ccicons}

\usepackage{pgfplots}
\usepgfplotslibrary{dateplot}

\usepackage{xspace}
\newcommand{\themename}{\textbf{\textsc{metropolis}}\xspace}

\title{Tecniche di animazione 3D nella realizzazione di un cortometraggio}
\subtitle{}												% L'obiettivo della mia tesi è quello di realizzare un cortometraggio animato in 3D
% \date{\today}
\date{10 Dicembre 2019}
\author{Leonardo Marini}
\institute{ALMA MATER STUDIORUM - UNIVERSITÀ DI BOLOGNA}
%\titlegraphic{\hfill\includegraphics[height=1.5cm]{logo.pdf}}

\begin{document}

\maketitle

\begin{frame}{Indice}																		% Si è quindi passati da una fase di analisi per capire quali fossero i requisiti.
  \setbeamertemplate{section in toc}[sections numbered] % Ad una di progettazione in cui sono si sono decise le diverse tecniche da usare poi nella produzione.
  \tableofcontents%[hideallsubsections]									% Le tecniche sono state quindi analizzate e studiate da un punto di vista principalmente matematico.
\end{frame}

\section[Intro]{Introduzione}

\begin{frame}{Analisi} 
      \begin{itemize}[<+- | alert@+>]								% Tra i requisiti abbiamo che il
        \item Cortometraggio animato in 3D 					% cortometraggio dev'essere realizzato in 3D con l'uso della computer grafica.
        \item Uso di diverse tecniche di animazione % Devono essere usate diverse tecniche di animazione 3D.
        \item Breve durata 													% Infine, dati i vincoli sulla durata di massimo 2 minuti, si è scelto di realizzare solo il trailer.
				\item Nessun requisito sulla storia 				% Non sono stati dati altri vincoli, per quanto riguarda la storia
      \end{itemize}
\end{frame}

\begin{frame}[fragile]{La storia}		% Per quest'ultima si è scelto di rappresentarne una ideata dal mio collega... spiegazione
  \begin{columns}[T,onlytextwidth]	% In essa vengono mostrati i tre protagonisti e il loro primo incontro. In cui i 2 capitani,
    \column{0.33\textwidth}					% ricercati in quanto ex pirati spaziali, si imbattono nel ragazzo e chiedono il suo aiuto per
      \begin{figure}								% scappare.
          \centering								% Il trailer mostrerà quindi alcune delle scene più salienti del corto, con sparatorie, e inseguimenti.
          \includegraphics[width=\textwidth]{figures/Capitano.png}
          \caption{Capitano}				% Questo ci ha permesso di concentrarci sulla fase di progettazione del corto, senza spendere tempo
      \end{figure}									% nell'ideazione della storia.

    \column{0.33\textwidth}
      \begin{figure}
          \centering
          \includegraphics[width=\textwidth]{figures/ragazzo.png}
          \caption{Ragazzo}
      \end{figure}

    \column{0.33\textwidth}
      \begin{figure}
          \centering
          \includegraphics[width=\textwidth]{figures/Capitana.png}
          \caption{Capitana}
      \end{figure}
  \end{columns}
\end{frame}

\section{Concetti di animazione}										% Come ho già detto, sono state usate diverse tecniche di animazione,
                                             				% Ne vengono quindi spiegati alcuni concetti.
\begin{frame}{Rappresentazioni di rotazione}				% Le rotazioni coprono un ruolo fondamentale nelle animazioni, ne sono state usate principalmente 3:
  \begin{columns}[T,onlytextwidth]									% Angoli di Eulero, Quaternioni e Rappresentazione matriciale.
    \column{0.33\textwidth}
    Angoli di Eulero
      \begin{itemize}[<+- | alert@+>]
							\item Concettualmente semplice							% La prima è quella concettualmente più semplice: utilizza 3 rotazioni una per ogni asse (X, Y e Z)
        \item Complessa e confusa in pratica							% L'implementazione però è complessa:
        \item L'ordine delle rotazioni è importante				% È importante stabilire l'ordine di queste tre rotazioni: cambiandolo cambia anche il risultato
        \item Gimbal lock e interpolazioni spezzate				% Inoltre, essendo gli assi dipendenti tra loro, questo modello presenta un problema, ovvero
      \end{itemize}										% che se due assi si allienano si perde un grado di libertà di rotazione. Questo problema è conosciuto come Gimbal Lock.

    \column{0.33\textwidth}
		Quaternioni
      \begin{itemize}[<+- | alert@+>]
        \item No gimbal lock												% I quaternioni risolvono il problema presente nella rapresentazione euleriana.
				\item Concettualmente complessa							% Il costo è che questo modello risulta concettualmente più complesso, infatti una rotazione completa consiste in un angolo di 720°
        \item Semplifica i calcoli									% Ha però il vantagigo aggiuntivo di semplificare i calcoli
        \item Interpolazioni consistenti e dirette  % e il risultato sono interpolazioni dirette, non spezzate sui 3 assi.
      \end{itemize}
      
    \column{0.33\textwidth}
    Matrici																			% In fine la rappresentazione con l'uso di matrici
      \begin{itemize}[<+- | alert@+>]						% permette di rappresentare qualsiasi tipo di trasformazione, non solo rotazioni.
				\item Qualsiasi tipo di trasformazione	% Sono concettualmente più complesse dei quaternioni, per questo motivo non vengono solitamente usate dall'utente finale (animatore)
				\item Parenting				% In blender ad esempio, esse sono utilizzate internamente, ogni rotazione viene convertita in matrice. Vengono anche usate per associazioni padre-figlio,
        \item Constraints			% vincoli
				\item Armature deform	% e deformazioni per l'uso di armature che permettono l'animazione di figure complesse come quella umana.
      \end{itemize}
  \end{columns}
\end{frame}

\begin{frame}{FK (Forward Kinematics)}					% Data la presenza di diversi personaggi, una delle tecniche maggiormente utilizzata è stata quella delle cinematiche,
		\begin{itemize}[<+- | alert@+>]							% che permettodo appunto di animare figure complesse, come quella umana.
    \item Figure complesse come quella umana    % ne esistono di due tipi. La cinematica diretta ha un approccio più naive ovvero, per posizionare un arto, viene applicata una rotazione
    \item Approccio naive												% ad ogni articolazione, partendo da quelle più vicine alla radice, che influenzano anche quelle più in basso nella gerarchia.
		\item Precisione del posizionamento					% Permette un posizionamento della figura molto preciso
    \item Difficile animare azioni comuni				% ma rende difficile animare azioni comuni siccome ruotare la spalla sposterebbe la posizione della mano.
  \end{itemize}
\end{frame}

\begin{frame}{IK (Inverse Kinematics)}					
		\begin{itemize}[<+- | alert@+>]							
    \item Approccio inverso											% La cinematica inversa ha un approccio opposto rispetto alla precedente.
    \item Figure complesse come quella umana		% Anch'essa, viene uilizzata per animare figure complesse ma,
		\item Semplifica le animazioni							% poiché viene posizionata diretamente l'ultima articolazione della gerarchia, e le altre vengono calcolte di conseguenza,
    \item Complessa da calcolare                % semplifica il lavoro dell'animatore, che non deve riposizionare la mano ogni volta che la spalla di un personaggio viene ruotata.
  \end{itemize}                                 % Ovviamente è più complessa da calcolare rispetto a quella diretta. Per farlo esistono diverse tecniche.
\end{frame}


\begin{frame}[fragile]{IK - Lo Jacobiano}				
  \begin{columns}[T,onlytextwidth]
    \column{0.5\textwidth}
    $
    \begin{bmatrix}
        \dfrac{\partial p_x}{\partial \theta_1} & \dfrac{\partial p_x}{\partial \theta_2} & \dots & \dfrac{\partial p_x}{\partial \theta_n} \\[2ex]
        \dfrac{\partial p_y}{\partial \theta_1} & \dfrac{\partial p_y}{\partial \theta_2} & \dots & \dfrac{\partial p_y}{\partial \theta_n} \\[2ex]
        \vdots & \vdots & \ddots & \vdots \\[2ex]
        \dfrac{\partial \alpha_z}{\partial \theta_1} & \dfrac{\partial \alpha_z}{\partial \theta_2} & \dots & \dfrac{\partial \alpha_z}{\partial \theta_n} 
    \end{bmatrix}
    $
    \column{0.5\textwidth}											% La maggior parte di esse utilizza lo Jacobiano, ovvero una matrice...
		Matrice di derivate parziali corrispondenti alla differenza della posizione attuale dell'end-effector rispetto alla posizione obiettivo.
		\begin{alertblock}{Proprietà}	
				\begin{itemize}													% questo permette di calcolare iterativamente la posizione dell'end-effector (mano)
				\item Soluzione iterativa								% che si avvicina alla posizione obiettivo finché non la raggiunge, se esiste una soluzione
				\item Simile al metodo del simplesso		% Il metodo in cui la soluzione è calcolata è quindi simile a quello del simplesso.
			\end{itemize}
    \end{alertblock}
	\end{columns}
\end{frame}


\section{Progettazione}
\begin{frame}{Rigging} % I rig dei modelli devono essere pensati in base alle animazioni da realizzare
  \begin{table}
					\caption{Diversi tipi di rig necessari un una figura umana in base ai compiti che deve eseguire}
		\begin{tabular}{lcr}
			\toprule
			Porzione del rig & Compito & Soluzione\\
			\midrule
			Braccia & raggiungere e gesticolare & IK e FK\\
			Mani & afferrare & FK\\
			Gambe & correre e camminare & IK\\
			\bottomrule
		\end{tabular}
	\end{table}	
\end{frame}

\section{Produzione}
\begin{frame}{Animazioni}
	\metroset{block=fill}
	\begin{columns}[T,onlytextwidth]

		\column{0.45\textwidth}
		\begin{exampleblock}{IK}
		camminata\\
		corsa\\
		raggiungere
		\end{exampleblock}
		\begin{exampleblock}{FK}
		raggiungere\\
		afferrare
		\end{exampleblock}

		\column{0.45\textwidth}
		\begin{exampleblock}{Curve}
		camminata\\
		corsa\\
		inseguimento spaziale
		\end{exampleblock}
		\begin{exampleblock}{Cicli}
		camminata\\
		corsa\\
		sparatorie
		\end{exampleblock}
	\end{columns}
\end{frame}

{\setbeamercolor{palette primary}{fg=orange}
\begin{frame}[standout]

Risultato

\end{frame}
}

\end{document}
